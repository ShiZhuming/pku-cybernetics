% Author:Zhuming Shi, Peking University
% Theme from https://github.com/matze/mtheme

\documentclass[12pt,AutoFakeBold,aspectratio=43,mathserif]{beamer}
\usepackage[english]{babel}

\usetheme{metropolis}
% \metroset{block=fill}
\usepackage{fontspec}% 控制字体
\setmainfont{Times New Roman}% 英文字体
\newfontfamily\arial{Arial}% Arial字体

\usepackage{xeCJK} % 中文支持
\setCJKmainfont{SimSun} % 中文字体
% \setCJKsansfont{SimHei}
\XeTeXlinebreaklocale "zh"%中文自动换行
\XeTeXlinebreakskip = 0pt plus 1pt%中文自动换行

\usepackage{graphicx}
\usepackage{subfigure}
\usepackage{caption}
\usepackage{hyperref}
\usepackage{amsthm,amsmath,amssymb,mathrsfs}% 数学符号和花体支持
\usepackage{booktabs}% 绘制三线表
\usepackage{latexsym}% 绘制特殊数学符号
\usepackage{siunitx}% 数学模式中使用SI单位

\usepackage[version=3]{mhchem}% 化学反应式
\usepackage{epstopdf}% 插入ChemDraw的.eps结构图

% 代码环境
\usepackage{listings}
\usepackage{color}

% \useoutertheme{infolines}

\definecolor{dkgreen}{rgb}{0,0.6,0}
\definecolor{gray}{rgb}{0.5,0.5,0.5}
\definecolor{mauve}{rgb}{0.58,0,0.82}

\lstset{frame=tb,
  language=c++,
  aboveskip=3mm,
  belowskip=3mm,
  showstringspaces=false,
  columns=flexible,
  basicstyle={\small\ttfamily},
  numbers=none,
  numberstyle=\tiny\color{gray},
  keywordstyle=\color{blue},
  commentstyle=\color{dkgreen},
  stringstyle=\color{mauve},
  breaklines=true,
  breakatwhitespace=true,
  tabsize=3
}

\setbeamerfont{footnote}{size=\tiny}

\newcommand{\unknow}[1]{{\arial \textbf{#1}}}%未知化合物格式
\newcommand{\substance}[1]{\textbf{\emph{#1}}}%矿物名称格式

% \setbeamertemplate{background}{\includegraphics[height=\paperheight]{figures/misaka1080.png}}

\makeatletter 
\renewcommand{\@thesubfigure}{\hskip\subfiglabelskip}
\makeatother

\title{从控制论到计算机}
\author{前沿第四组}
\date{10月22日}

\begin{document}
    \begin{frame}
        % \frametitle{}
        \titlepage
    
    \end{frame}
    \frame{\frametitle{Outline}\tableofcontents[hideallsubsections]}
    
    \section{机构与变异度}
    	\begin{frame}{控制论}
    		\begin{itemize}
    			\item 控制论的背景:起源于二战时期对飞机的防御和打击
    			\item 控制论的诞生:1948 年,维纳的《控制论:或关于在动物和机器中控制和通信的科学》出版,标志着这个跨学科方向正式诞生。
            \end{itemize}
            \begin{figure}
                \centering
                \includegraphics[width=.3\paperwidth]{figures/figure1_1.jpg}
            \end{figure}
    	    
    		\end{frame}
    	\begin{frame}
    		1956 年,阿什比出版《控制论导论》,对控制论这个学科进行了更具一般性的总结和发展,明确提出控制论“本质上是关于机器的功能和行为”,“控制论将‘所有可能的机器’视为自己的研究题材”,正如几何学将所有可能的形状作为研究题材,而不是特定的圆球或方块。
    		
    		
    		我们的小组展示,通过这本《控制论导论》,对控制论进行一次管中窥豹。
    		
    	\end{frame}
    	
    	\begin{frame}{前言}
    		什么是控制论?
    		
    		维纳:(关于)动物和机器中控制和通信的科学
    		
    		Or: 它是研究这样一类系统的科学,在这类系统中能量无关紧要,而信息及控制却非常重要。
    		
    		\end{frame}
    	\begin{frame}{控制论的用处}
    		\begin{itemize}
    			\item 一套统一的词汇和概念
    			
    			\item 对复杂的系统,控制论给出统一研究方法
    
    		\end{itemize}
    	\end{frame}
    	\begin{frame}{基本定义}
	    		% \begin{block}
    			“我们认为某一事物所具有的种种性质,归根到底无非是给它的行为起些名字”-Horrick
    		    % \end{block}
    	
    		\begin{itemize}
    			\item 差异
    			\item 变换
    			\item 封闭性
    			\item 单值的
    			\item 一一对应
    			\item 多一变换
    		\end{itemize}
    	\end{frame}
    	\begin{frame}{变换的表示}
    		$T(n)$表示一个变换
    		
    		研究一个原象在多次重复变换下的结果,用有向图的形式表示。
            \begin{figure}
                \centering
                \includegraphics[width=0.6\textwidth]{figures/figure1_2.jpg}
            \end{figure}
    		\end{frame}
    	\begin{frame}{确定性机器}
    		把行为与封闭单值变换相同的机器定义为确定性机器
    		
    		我们不关心算子具体是什么,而只关心确定的变换,它只与变化的事实有关,而不涉及那些带有假设性的原因
    		
    		机器与动态图所表示的变换完全对应时,则称变换是机器的标准表达式,机器是变换的具体化
    		
    		任何现实的确定性机器或能动系统都对应一个封闭的单值变换
    	\end{frame}
    	\begin{frame}{有输入的机器}
    		对于同一种机器,从一种状态变到另一种状态的是机器的性能,而从一种变换到另一种变换的变化是性能的变化,这取决于输入的参数。这种机器称为有输入的机器或\textbf{变换器}。
    		\end{frame}
    	\begin{frame}{耦合与反馈}
    		\begin{block}{耦合}
    			将机器结合到一起,使每台机器对其他机器的影响只限于改变后者的输入,而不改变其性能(即变换)
    			\end{block}
    		将P与Q耦合,使P对Q有影响而Q对P无影响,称为P主制Q
    		
    		当P与Q相互影响时,称该系统有反馈
    		
    		\end{frame}
    	\begin{frame}{稳定性}
    		不变量:平衡点(不动点),循环圈,稳定域
    		\begin{figure}
                \subfigure[]{
                    \includegraphics[height=.33\paperheight]{figures/figure1_3.jpg}
                }
                \subfigure[]{
                    \includegraphics[height=.33\paperheight]{figures/figure1_4.jpg}
                }
            \end{figure}
    		干扰,稳定平衡,不稳定平衡,随遇平衡
    		\end{frame}
    	\begin{frame}{特大系统与黑箱}
    		特大系统的概念
    		黑箱的概念
    		研究黑箱的方法:对黑箱进行不同的输入,记录黑箱的输出,变成一串含两个分量(输入、输出)的矢量表,称为登记表。
    	\end{frame}
    	\begin{frame}{同构与同态}
    		两个机器的标准表达式,如果存在一一变换,能将一个机器的状态(输入与输出)变为另一机器的状态,同时把一种表示式变为另一种表示式的,称两个机器同构。
    		
    		
    		\textbf{同态}:对于两个机器,存在多一变换,使得一个机器经过变换后与另一机器同构,则称较简单的那个机器是前者的同态象。
    		
    		\end{frame}
		\begin{frame}{变异度}
			\begin{itemize}
				\item 衡量机器可能状态的参数,常用对数表示;
				\item 状态是多种多样的,状态空间与基本事件空间类似;
				\item 矢量状态的变异度不会大于其每个分量的变异度之和; 对于单值变换的机器,变异度不会增加。
			\end{itemize}
		\end{frame}
		\begin{frame}{机器与变异度的传输}
			\quad 机器是信息传输的载体,信息传输伴随着变异度的改变。例如,对下图的机器:
			\begin{figure}
                \centering
                \includegraphics[width=.4\textwidth]{figures/figure2_1.PNG}
            \end{figure}
			\quad 初始状态为A, 对QR编码得到CB, 对CB译码得到QR;但对BC就不能译码
		\end{frame}
		\begin{frame}{多值变换下的变异度}
			\quad 推广单值变换的属性,使机器能从一个状态按照不同概率转移到其它状态, 变异度定义为事件的信息熵。
			
			\quad $\text{变异度} := - \sum_i p_i \log p_i$
			
			\quad 可以看到,之前的变异度是在每个状态等可能出现时的特殊情况。
			
			\quad $I_0 := - \sum_{i = 1}^n \frac{1}{n}\log\frac{1}{n} = \log n$
			
		\end{frame}
		\begin{frame}{例:马尔可夫链}
			\begin{block}{马尔可夫链}
				指一类事件,当前时刻要发生的事件只依赖于当前状态。
			\end{block}
			
			\quad 马氏链中,变异度定义为以信息熵按照稳定状态时各个状态的概率加权平均
            \begin{figure}
                \centering
                \includegraphics[width =.4\textwidth]{figures/figure2_2.PNG}
            \end{figure}
			$I = 0.811 * 0.449 + 1.061 * 0.429 + 1.061 * 0.122 = 0.842bit$
			
		 	\quad 以变异度为媒介,不同事物的不稳定程度可以相互比较
			
		\end{frame}

    \section{调节与控制}

    \begin{frame}{浅谈调节}
        \begin{itemize}
                \item 调节作用堵塞了干扰源传向基本变量的变异度
                \item 例:恒温淋浴设备
               \begin{figure}[H]
                \centering
                \includegraphics[width=.6\textwidth]{figures/pic1.png}
                \end{figure}
        \end{itemize}
        
    \end{frame}
    \begin{frame}{必须变异度}
        \begin{itemize}
              \item D和R进行游戏, R在D之后做动作
              \begin{figure}[H]
                \centering
                \includegraphics[width=.4\textwidth]{figures/pic2.png}
                \end{figure}
                \item 结局的变异度不能小于$\frac{D\text{的变异度}}{R\text{的变异度}}=\frac{9}{3}$
                \item 只有变异度才能消灭变异度!
        \end{itemize}
    \end{frame}
    \begin{frame}{必须变异度率}
        \begin{itemize}
                \item 若调节器R已给定, 则结局E的熵不小于干扰D的熵
                \item $H_R(E)\geq H_R(D)$
                \item 其他附加条件(如噪声、复合干扰、调节的误差等)都可以视作R的一部分
        \end{itemize}
    \end{frame}
    \begin{frame}{马尔可夫型机器}
        \begin{itemize}
                \item 非确定性机器: 更加曲折但更加鲁棒地趋向平衡状态
                \begin{figure}[H]
                \centering
                \includegraphics[width=.6\textwidth]{figures/pic3.png}
                \end{figure}
                \item 例: 捕蝇纸对于房间中苍蝇的作用
        \end{itemize}
        
    \end{frame}
      
    \begin{frame}{其他调节}
        \begin{itemize}
                \item 特大系统: 系统T相对于调节器R来说很大, 怎么办 ?
                \item[1] 约束
                \begin{figure}[H]
                \centering
                \includegraphics[width=.6\textwidth]{figures/pic4.png}
                \end{figure}
                \item[2] 关注重复干扰的总结果
                \item[3] 功率放大器
        \end{itemize}
    \end{frame}

    \section{从控制论到计算机}
    \subsection{控制科学(学科)\& 计算机}
    \begin{frame}
        \frametitle{控制科学(学科)\& 计算机}
        \pause
        控制科学和计算机是两门学科. \\
        在隔壁,研究控制科学的学科是「自动化」。
        \begin{figure}[htbp]
            \caption{清华大学自动化专业本科课程}
            \setlength{\abovecaptionskip}{0.cm}
            \setlength{\belowcaptionskip}{-0.cm}
            \centering
            \vspace{-0.3cm}
            \setlength{\abovecaptionskip}{0.cm}
            \setlength{\belowcaptionskip}{-0.cm}
            \includegraphics[width=0.9\textwidth]{figures/3-1.jpg}
        \end{figure}
    \end{frame}
    \subsection{控制论 \& 控制理论}
    \begin{frame}
        \frametitle{控制论 \& 控制理论}
        \begin{block}{\textnormal{「控制论」与「控制理论」是一回事吗?}}
        \end{block} \pause
        控制论(Cybernetics)与控制理论(Control Theory)是两个不同的概念。 \\
        控制论将控制系统作为一个在整体概念进行研究,而控制理论着重于信息因素,研究系统中各部分的相互作用以及系统的结构。
        \pause
        
    
    \end{frame}
    \subsection{控制论 \& 计算机}
    \begin{frame}
        \frametitle{控制论和计算机的关系}
        \begin{block}{\textnormal{为什么控制论和计算机有关系?}} \end{block} \pause
        \begin{columns}
            \begin{column}{.5\linewidth}
                \begin{itemize}
                    \item  计算机$\to$控制论:新方向,新思路。
                    \item  控制(理)论$\to$计算机:并行、动态分支预测、可编程控制器
                \end{itemize}
            \end{column}
            \begin{column}{.5\linewidth}
                \includegraphics[width=.4\paperwidth]{figures/3-3.jpg}
            \end{column}
        \end{columns}
        
    \end{frame}
    \begin{frame}
        \frametitle{强化学习 \& 最优控制}
        强化学习和控制理论有着很深的联系。 \pause
        \begin{itemize}
            \item  都是研究利用过去的信息来强化未来操纵的动态系统。\pause
            \item  目的都是设计一个系统,其能够使用高度结构化的感知信息,做出规划和控制以适应环境变化,同时在遇到新场景时做好保障。因此,可以使用强化学习的思想和算法来解决控制系统的问题。 
        \end{itemize}
    \end{frame}
    \begin{frame}
        \frametitle{强化学习 \& 最优控制}
        \begin{columns}
            \begin{column}{.4\linewidth}
                (上图是强化学习,下图是控制器,线的颜色相同的部分是对应的关系)
            \end{column}
            \begin{column}{.5\linewidth}
                \includegraphics[width=.5\paperwidth]{figures/rl_for_control_systems.png}
            \end{column}
        \end{columns}
    \end{frame}
    \begin{frame}
        \frametitle{强化学习 \& 最优控制}
        以最基本的线性二次型控制器为例:\footnote{\url{https://www.zhihu.com/question/401591393/answer/1285670063}}
        \begin{figure}[htbp]
            \setlength{\abovecaptionskip}{0.cm}
            \setlength{\belowcaptionskip}{-0.cm}
            \centering
            \vspace{-0.3cm}
            \setlength{\abovecaptionskip}{0.cm}
            \setlength{\belowcaptionskip}{-0.cm}
            \includegraphics[width=0.7\textwidth]{figures/3-5.jpg}
        \end{figure}
        
    \end{frame}
    \subsection{控制论应用举例}
    \begin{frame}
        \frametitle{控制论应用举例}
        接下来是控制论应用的一些例子。
    \end{frame}
    \begin{frame}
        \frametitle{网络控制}
        \footnotesize 网络控制是一个很大的领域,涉及许多主题,包括路由、数据缓存和电源管理。这些控制问题的一些特点使它们非常具有挑战性: \pause
        \begin{itemize}
            \item  系统的超大规模:Internet可能是人类所建立的最大的反馈控制系统。
            \item  控制问题的分散化本质:必须快速做出局部决策,并且仅基于局部信息。
            \item  其它:比如对服务质量的不同要求等。
        \end{itemize} \pause
        网络控制下一阶段将涉及更多的物理环境和对网络控制的增加使用,需要通信、计算和控制的融合。 \\ \pause
        另一个可能的发展方向:目前的网络控制系统几乎普遍基于同步、定时系统来避免数据丢失,我们是否可以开发一个理论和实践控制系统,在一个分布式的、异步的、基于分组的环境中运行,这将在许多情景下更好地适应我们的需求。
    \end{frame}
    \begin{frame}
        \frametitle{生物控制}
        \footnotesize 生物学正变得越来越容易被工程中常用的方法所使用:数学建模、系统理论、计算和合成的抽象方法。控制原理是生物工程中许多关键问题的核心,并将在该领域的未来发挥作用。下图就是一个生物控制网络逆向(并最终向前推进)工程。 \pause
        \begin{figure}[htbp]
            \setlength{\abovecaptionskip}{0.cm}
            \setlength{\belowcaptionskip}{-0.cm}
            \centering
            \vspace{-0.3cm}
            \setlength{\abovecaptionskip}{0.cm}
            \setlength{\belowcaptionskip}{-0.cm}
            \includegraphics[width=0.6\textwidth]{figures/3-4.png}
        \end{figure}
    \end{frame}
    \section*{致谢}

    \begin{frame}
        \frametitle{致谢}
        \begin{columns}
            \begin{column}{.5\linewidth}
                我们的团队(排名不分先后):

                王泽州\quad 金皓宇\quad 陈齐治
        
                陈思元\quad 李鸿泽\quad 赵晨琪
                
                邓朝萌\quad 谭开云\quad 施朱鸣
            
                \bigskip

                感谢老师们和助教们的帮助!

                祝大家期中顺利,谢谢聆听!
            \end{column}
            \begin{column}{.5\linewidth}
                \includegraphics[width=.4\paperwidth]{figures/misaka558.png}
            \end{column}
        \end{columns}
        *\footnote{组长邮箱:\href{mailto:shizhuming@pku.edu.cn}{shizhuming@pku.edu.cn} \\ LaTeX代码开源在\url{https://github.com/ShiZhuming/pku-cybernetics}}
    \end{frame}
    
\end{document}